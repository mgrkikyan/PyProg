\documentclass[a4paper]{article}
\usepackage[14pt]{extsizes}
\usepackage[utf8]{inputenc}
\usepackage[russian]{babel}
\babelfont{rm}{Lato}
\usepackage[left=20mm, top=15mm, right=15mm, bottom=15mm, nohead, footskip=10mm]{geometry}
 
\begin{document}
\thispagestyle{empty}
\begin{center}

\large{Бюджетное учреждение высшего образования Ханты-Мансийского автономного округа – Югры}\\
\large{«СУРГУТСКИЙ ГОСУДАРСТВЕННЫЙ УНИВЕРСИТЕТ»}\\

\hfill \break
\normalsize{Политехнический институт}\\
 \hfill \break
\normalsize{Кафедра прикладной математики}\\
\hfill\break
\hfill \break
\hfill \break
\hfill \break
\large{Лабораторная работа №5}\\
\normalsize{Тема: Построение графиков в Python}\\
\hfill \break
\hfill \break
\hfill \break
\hfill \break
Дисциплина «Программирование»\\
\hfill \break
\small{Направление 01.03.02 «Прикладная математика и информатика»\\
Направленность (профиль): «Технологии программирования и анализ данных»}\\
\end{center}

\hfill \break
\hfill \break
\hfill \break
\hfill \break
\hfill \break
\hfill \break

\begin{flushright}
\normalsize{ 
\begin{tabular}{ccrr}
 & & & Преподаватель: Бычин Игорь Валерьевич \\\\
 & & & Преподаватель \\\\
% Должность, & \underline{\hspace{3cm}} &  степень & Иванов И.И. \\\\
 & & & Студент гр. №601-31 \\\\
 & & & Гркикян М.Э. \\\\
\end{tabular}
}\\
\end{flushright}

\hfill \break
\hfill \break
\begin{center} Сургут \the\year{} г.  \end{center}
\end{document}